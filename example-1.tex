%!TEX program = xelatex
\title          {Logic in Computer Science(Homework 1)}
\author         {10175101148 Junjie Chen}

\documentclass{article}
\usepackage{boxproof}
\usepackage{a4wide}
% \usepackage{daymonthyear}

\def\meta#1{\mbox{$\langle\hbox{#1}\rangle$}}
\def\macrowitharg#1#2{{\tt\string#1\bra\meta{#2}\ket}}

{\escapechar-1 \xdef\bra{\string\{}\xdef\ket{\string\}}}

\showboxbreadth 999
\showboxdepth 999
\tracingoutput 1


\let\imp\to

\def\premise{\mathrm{premise}}
\def\assumption{\mathrm{assumption}}
\def\MT{\mathrm{MT\ }}
\def\LEM{\mathrm{LEM}}
\def\intro{\mathrm{i\ }}
\def\elim{\mathrm{e\ }}
\def\introa{\mathrm{i_1\ }}
\def\elima{\mathrm{e_1\ }}
\def\introb{\mathrm{i_2\ }}
\def\elimb{\mathrm{e_2\ }}

\def\lt{<}
\def\eqdef{=}
\def\eps{\mathrel{\epsilon}}
\def\biimplies{\leftrightarrow}
\def\flt#1{\mathrel{{#1}^\flat}}
\def\setof#1{{\left\{{#1}\right\}}}
\let\implies\to
\def\KK{{\mathsf K}}
\let\squashmuskip\relax

%=======================================================================
\begin{document}
\maketitle
%=======================================================================
\section{Prove the following Theroems with nature deduction.}
\subsection{\(\neg(p \wedge q) \dashv\vdash \neg q \vee \neg p\)}
Proof:\\
Left to right:
$$
\begin{proofbox}
   \: \neg (p\land q) \= \premise\\
   \: p\vee\neg p \= \LEM \\
      \[
         \: p \= \assumption\\
         \[
            \: q \= \assumption\\
            \: p \land q \= \land\intro3,4\\
            \: \bot \= \neg\elim1,5\\
         \]
         \: \neg q \= \neg\intro4-6\\
         \: \neg q \vee \neg p \= \vee\introa 7\\
      \]
      \[
         \: \neg p \= \assumption\\
         \: \neg q \vee \neg p \= \vee\introb 9\\
      \]
      \: \neg q \vee \neg p \= \vee\elim 2,3-8,9-10\\
\end{proofbox}$$
Right to left:
$$
\begin{proofbox}
   \: \neg q \vee \neg p \= \premise\\
      \[
         \: \neg q \= \assumption\\
         \[
            \: p\land q \= \assumption\\
            \: q \= \land\elimb 3\\
            \: \bot \= \neg\elim2,4\\
         \]
         \: \neg (p\land q) \= \neg\intro 3-5\\
      \]
      \[
         \: \neg p \= \assumption\\
         \[
            \: p\land q \= \assumption\\
            \: p \= \land\elima 8\\
            \: \bot \= \neg\elim 7,9\\
         \]
         \: \neg (p\land q) \= \neg\intro 8-10\\
      \]
      \: \neg (p\land q) \= \vee\elim 1,2-6,7-11\\
\end{proofbox}$$
So \(\neg(p \wedge q) \dashv\vdash \neg q \vee \neg p\).

\subsection{\(p \rightarrow q \dashv\vdash \neg q \rightarrow \neg p\)}
Proof:\\
Left to right:
$$
% \proofboxformulawidth=18em\relax
\begin{proofbox}
   \: p\to q \= \premise\\
      \[
         \: \neg q \= \assumption\\
         \: \neg p \=  \MT1,2\\
      \]
   \: \neg q\to\neg p \= \to\intro2-3\\
\end{proofbox}$$
Right to left:
$$
\begin{proofbox}
   \: \neg q\to\neg p \= \premise\\
      \[
         \: p \= \assumption\\
         \: \neg\neg p \= \neg\neg\intro2\\
         \: \neg\neg q \=  \MT1,3\\
         \: q \= \neg\neg\elim4\\
      \]
   \: p\to q \= \to\intro2-5\\
\end{proofbox}$$
So \(p \rightarrow q \dashv\vdash \neg q \rightarrow \neg p\).
\subsection{\(p \wedge q \rightarrow p \dashv\vdash r \vee \neg r\)}
Proof:\\
Left to right:
$$
% \proofboxformulawidth=18em\relax
\begin{proofbox}
   \: r\vee\neg r \= \LEM
\end{proofbox}$$
Right to left:
$$
\begin{proofbox}
   \[
      \: p\land q \= \assumption\\
      \: p \= \land\elima1\\
   \] 
\: p\land q\to p \= \to\intro1-2\\
\end{proofbox}$$
So \(p \wedge q \rightarrow p \dashv\vdash r \vee \neg r\).

% page 1: (a)

% \begin{proofbox}
%    \: P\land Q \\
%    \: P         \= \elim\land\\
% \end{proofbox}

% \begin{proofbox}
%    \lbl{1}\: P \\
%    \[
%       \: Q \\
%       \: P \= (\ref{1}) \\
%    \]
%    \: Q\to P \= \intro\to \\
% \end{proofbox}


% page 2: (c)

% \begin{proofbox}
%    \lbl{2}\: P\\
%    \[
%       \lbl{3}\: Q\\
%       \: P\land Q\= \intro\land(\ref{2},\ref{3}) \\
%    \]
%    \: Q\to(P\land Q) \= \intro\to \\
% \end{proofbox}

% (g)

% \begin{proofbox}
%    \lbl{6}\: P\to(Q\to R)\\
%    \[
%       \lbl{4}\: P\to Q\\
%       \[
%          \lbl{5}\: P\\
%          \lbl{8}\: Q \= \elim\to(\ref{4},\ref{5})\\
%          \lbl{7}\: Q\to R \= \elim\to(\ref{6},\ref{5})\\
%          \: R \= \elim\to(\ref{7},\ref{8})\\
%       \]
%       \: P\to R \= \intro\to \\
%    \]
%    \: (P\to Q)\to(P\to R) \= \intro\to \\
% \end{proofbox}

% page 3: (h)

% \begin{proofbox}
%    \lbl{10}\: P\to(Q\to R)\\
%    \[
%       \lbl{9}\: P\land Q\\
%       \lbl{11}\: P \= \elim\land1(\ref{9})\\
%       \lbl{12}\: Q\to R \= \elim\to(\ref{10},\ref{11})\\
%       \lbl{13}\: Q \=\elim\land2(\ref{9})\\
%       \: R \=\elim\to(\ref{12},\ref{13})\\
%    \]
%       \: P\land Q\to R \= \intro\to \\
% \end{proofbox}

% (i)

% \begin{proofbox}
%    \lbl{16}\: P\land Q\to R \\
%    \[
%       \lbl{14}\: P \\
%       \[
%          \lbl{15}\: Q \\
%          \lbl{17}\: P\land Q \= \intro\land(\ref{14},\ref{15})\\
%          \: R \= \elim\to(\ref{16},\ref{17})
%       \]
%       \: Q\to R \= \intro\to \\
%    \]
%    \: P\to(Q\to R) \= \intro\to \\
% \end{proofbox}

% (j)

% \begin{proofbox}
%    \lbl{18}\: P\to Q\\
%    \lbl{20}\: \lnot Q\\
%    \[
%       \lbl{19}\: P \\
%       \lbl{21}\: Q \= \elim\to(\ref{18},\ref{19})\\
%       \: \bot \= \elim\lnot(\ref{20},\ref{21})\\
%    \]
%    \: \lnot P \= \intro\lnot \\
% \end{proofbox}

% page 5: (k)

% \begin{proofbox}
%    \lbl{22}\: \lnot P\\
%    \[
%       \lbl{23}\: P\\
%       \[
%          \: \lnot Q\\
%          \: \lnot P \land P \= \intro\land(\ref{22},\ref{23})\\
%          \: \bot \= \elim\lnot(\ref{22},\ref{23})\\
%       \]
%       \: \lnot\lnot Q \= \intro\lnot\\
%       \: Q \= \lnot\lnot\\
%    \]
%    \: P\to Q \= \intro\to \\
% \end{proofbox}

% page 8: (o)

% \begin{proofbox}
%    \lbl{24}\: P\to Q \\
%    \[
%       \lbl{26}\: \lnot Q \\
%       \[
%          \lbl{25}\: P \\
%          \lbl{27}\: Q \= \elim\to(\ref{24},\ref{25})\\
%          \: \bot \= \elim\lnot(\ref{26},\ref{27})\\
%       \]
%       \: \lnot P \= \intro\lnot\\
%    \]
%    \: \lnot Q\to \lnot P \= \intro\to \\
% \end{proofbox}

% (p) The $\lnot\lnot$ rule is unnecessary!

% \begin{proofbox}
%    \lbl{28}\: P\to Q \\
%    \[
%       \: \lnot\lnot P\\
%       \lbl{29}\: P \= \lnot\lnot\\
%       \[
%          \lbl{31}\: \lnot Q \\
%          \lbl{30}\: Q \= \elim\to(\ref{28},\ref{29})\\
%          \: Q\land\lnot Q \= \intro\land(\ref{30},\ref{31})\\
%          \: \bot \= \elim\lnot\\
%       \]
%       \: \lnot\lnot Q \= \intro\lnot\\
%    \]
%    \: \lnot\lnot P\to\lnot\lnot Q \= \intro\to \\
% \end{proofbox}
%=======================================================================
\end{document}


