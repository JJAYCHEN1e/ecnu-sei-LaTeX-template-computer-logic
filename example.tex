%!TEX program = xelatex
\title          {Logic in Computer Science - Example}
\author         {JJAYCHEN}

\documentclass{article}
\usepackage{boxproof}
\usepackage{a4wide}
% \usepackage{daymonthyear}

\def\meta#1{\mbox{$\langle\hbox{#1}\rangle$}}
\def\macrowitharg#1#2{{\tt\string#1\bra\meta{#2}\ket}}

{\escapechar-1 \xdef\bra{\string\{}\xdef\ket{\string\}}}

\showboxbreadth 999
\showboxdepth 999
\tracingoutput 1


\let\imp\to

\def\premise{\mathrm{premise}}
\def\assumption{\mathrm{assumption}}
\def\MT{\mathrm{MT\ }}
\def\LEM{\mathrm{LEM}}
\def\intro{\mathrm{i\ }}
\def\elim{\mathrm{e\ }}
\def\introa{\mathrm{i_1\ }}
\def\elima{\mathrm{e_1\ }}
\def\introb{\mathrm{i_2\ }}
\def\elimb{\mathrm{e_2\ }}

\def\lt{<}
\def\eqdef{=}
\def\eps{\mathrel{\epsilon}}
\def\biimplies{\leftrightarrow}
\def\flt#1{\mathrel{{#1}^\flat}}
\def\setof#1{{\left\{{#1}\right\}}}
\let\implies\to
\def\KK{{\mathsf K}}
\let\squashmuskip\relax

%=======================================================================
\begin{document}
\maketitle
%=======================================================================
\section{Prove the following Theroems with nature deduction.}
\subsection{\(\neg(p \wedge q) \dashv\vdash \neg q \vee \neg p\)}
Proof:\\
Left to right:
$$
\begin{proofbox}
   \: \neg (p\land q) \= \premise\\
   \: p\vee\neg p \= \LEM \\
      \[
         \: p \= \assumption\\
         \[
            \: q \= \assumption\\
            \: p \land q \= \land\intro3,4\\
            \: \bot \= \neg\elim1,5\\
         \]
         \: \neg q \= \neg\intro4-6\\
         \: \neg q \vee \neg p \= \vee\introa 7\\
      \]
      \[
         \: \neg p \= \assumption\\
         \: \neg q \vee \neg p \= \vee\introb 9\\
      \]
      \: \neg q \vee \neg p \= \vee\elim 2,3-8,9-10\\
\end{proofbox}$$
Right to left:
$$
\begin{proofbox}
   \: \neg q \vee \neg p \= \premise\\
      \[
         \: \neg q \= \assumption\\
         \[
            \: p\land q \= \assumption\\
            \: q \= \land\elimb 3\\
            \: \bot \= \neg\elim2,4\\
         \]
         \: \neg (p\land q) \= \neg\intro 3-5\\
      \]
      \[
         \: \neg p \= \assumption\\
         \[
            \: p\land q \= \assumption\\
            \: p \= \land\elima 8\\
            \: \bot \= \neg\elim 7,9\\
         \]
         \: \neg (p\land q) \= \neg\intro 8-10\\
      \]
      \: \neg (p\land q) \= \vee\elim 1,2-6,7-11\\
\end{proofbox}$$
So \(\neg(p \wedge q) \dashv\vdash \neg q \vee \neg p\).

\subsection{\(p \rightarrow q \dashv\vdash \neg q \rightarrow \neg p\)}
Proof:\\
Left to right:
$$
% \proofboxformulawidth=18em\relax
\begin{proofbox}
   \: p\to q \= \premise\\
      \[
         \: \neg q \= \assumption\\
         \: \neg p \=  \MT1,2\\
      \]
   \: \neg q\to\neg p \= \to\intro2-3\\
\end{proofbox}$$
Right to left:
$$
\begin{proofbox}
   \: \neg q\to\neg p \= \premise\\
      \[
         \: p \= \assumption\\
         \: \neg\neg p \= \neg\neg\intro2\\
         \: \neg\neg q \=  \MT1,3\\
         \: q \= \neg\neg\elim4\\
      \]
   \: p\to q \= \to\intro2-5\\
\end{proofbox}$$
So \(p \rightarrow q \dashv\vdash \neg q \rightarrow \neg p\).

\subsection{\(p \wedge q \rightarrow p \dashv\vdash r \vee \neg r\)}
Proof:\\
Left to right:
$$
% \proofboxformulawidth=18em\relax
\begin{proofbox}
   \: r\vee\neg r \= \LEM
\end{proofbox}$$
Right to left:
$$
\begin{proofbox}
   \[
      \: p\land q \= \assumption\\
      \: p \= \land\elima1\\
   \] 
\: p\land q\to p \= \to\intro1-2\\
\end{proofbox}$$
So \(p \wedge q \rightarrow p \dashv\vdash r \vee \neg r\).

\section{Prove the following theroems with deduction rules.}
\subsection{$\forall x(P(x) \rightarrow \neg Q(x)) \vdash \neg(\exists x(P(x) \wedge Q(x)))$}
Proof:\\
$$
\begin{proofbox}
   \: \forall x(P(x)\rightarrow \neg Q(x)) \= \premise\\
   \[
      \: \exists x(P(x)\wedge Q(x)) \= \assumption\\
      \[
         x_0 \: P(x_0)\wedge Q(x_0) \= \assumption\\
         \: P(x_0) \= \wedge\elima 3\\
         \: Q(x_0) \= \wedge\elimb 3\\
         \: P(x_0) \rightarrow \neg Q(x_0) \= \forall x\ \elim 1\\
         \: \neg Q(x_0) \= \rightarrow\elim 4,6\\
         \: \bot \= \neg \elim 5,7
      \]
      \: \bot \= \exists x\ \elim 2,3-8\\
   \]
   \: \neg(\exists(P(x)\wedge Q(x))) \=\neg\intro 2-9 
\end{proofbox}$$
\subsection{$\forall x(P(x) \leftrightarrow x=b) \vdash P(b) \wedge \forall x \forall y(P(x) \wedge P(y) \rightarrow x=y)$}
Proof:\\
$$
\begin{proofbox}
   \: \forall x(P(x) \leftrightarrow x=b) \= \premise\\
   \: b=b \= =\intro\\
   \: b=b\rightarrow P(b) \= \forall x\ \elim 1\\
   \: P(b) \= \rightarrow\elim 2,3
   \[
      x_0\: P(x_0)\rightarrow x_0=b \= \forall x\ \elim 1\\
      \[
         y_0 \: P(y_0)\rightarrow y_0=b \= \forall x\ \elim 1\\
         \[
            \: P(x_0) \wedge P(y_0) \= \assumption \\
            \: P(x_0) \= \wedge\elima 7\\
            \: P(y_0) \= \wedge\elimb 7\\
            \: x_0=b \= \rightarrow\elim 5,8\\
            \: y_0=b \= \rightarrow\elim 6,9\\
            \: x_0=y_0 \= =\elim 10,11
         \]
         \: P(x_0)\wedge P(y_0) \rightarrow  x_0=y_0 \= \rightarrow\intro 7-12
      \]
      \: \forall y (P(x_0)\wedge P(y)\rightarrow x_0=y) \= \forall y\ \intro 6-13
   \]
   \: \forall x\forall y (P(x)\wedge P(y)\rightarrow x=y) \= \forall x\ \intro 5-14\\
   \: P(b) \wedge \forall x \forall y(P(x) \wedge P(y) \rightarrow x=y) \= \wedge\intro 4,15
\end{proofbox}$$

\section{Prove the following theroems with deduction rules.}
\subsection{$\forall x(P(x) \rightarrow \neg Q(x)) \vdash \neg(\exists x(P(x) \wedge Q(x)))$}
Proof:\\
$$
\begin{proofbox}
   \: \forall x(P(x)\rightarrow \neg Q(x)) \= \premise\\
   \[
      \: \exists x(P(x)\wedge Q(x)) \= \assumption\\
      \[
         x_0 \: P(x_0)\wedge Q(x_0) \= \assumption\\
         \: P(x_0) \= \wedge\elima 3\\
         \: Q(x_0) \= \wedge\elimb 3\\
         \: P(x_0) \rightarrow \neg Q(x_0) \= \forall x\ \elim 1\\
         \: \neg Q(x_0) \= \rightarrow\elim 4,6\\
         \: \bot \= \neg \elim 5,7
      \]
      \: \bot \= \exists x\ \elim 2,3-8\\
   \]
   \: \neg(\exists(P(x)\wedge Q(x))) \=\neg\intro 2-9 
\end{proofbox}$$
\subsection{$\forall x(P(x) \leftrightarrow x=b) \vdash P(b) \wedge \forall x \forall y(P(x) \wedge P(y) \rightarrow x=y)$}
Proof:\\
$$
\begin{proofbox}
   \: \forall x(P(x) \leftrightarrow x=b) \= \premise\\
   \: b=b \= =\intro\\
   \: b=b\rightarrow P(b) \= \forall x\ \elim 1\\
   \: P(b) \= \rightarrow\elim 2,3
   \[
      x_0\: P(x_0)\rightarrow x_0=b \= \forall x\ \elim 1\\
      \[
         y_0 \: P(y_0)\rightarrow y_0=b \= \forall x\ \elim 1\\
         \[
            \: P(x_0) \wedge P(y_0) \= \assumption \\
            \: P(x_0) \= \wedge\elima 7\\
            \: P(y_0) \= \wedge\elimb 7\\
            \: x_0=b \= \rightarrow\elim 5,8\\
            \: y_0=b \= \rightarrow\elim 6,9\\
            \: x_0=y_0 \= =\elim 10,11
         \]
         \: P(x_0)\wedge P(y_0) \rightarrow  x_0=y_0 \= \rightarrow\intro 7-12
      \]
      \: \forall y (P(x_0)\wedge P(y)\rightarrow x_0=y) \= \forall y\ \intro 6-13
   \]
   \: \forall x\forall y (P(x)\wedge P(y)\rightarrow x=y) \= \forall x\ \intro 5-14\\
   \: P(b) \wedge \forall x \forall y(P(x) \wedge P(y) \rightarrow x=y) \= \wedge\intro 4,15
\end{proofbox}$$

\end{document}


